\documentclass[12pt, a4]{article}
\usepackage[english]{babel}
\usepackage[utf8x]{inputenc}
\usepackage{fullpage}
\usepackage{listings}
\usepackage{graphicx}
\usepackage{color}

%Syntax highlighting
\definecolor{blue-violet}{rgb}{0.54, 0.17, 0.89}
\definecolor{ao}{rgb}{0.0, 0.5, 0.0}
\definecolor{amaranth}{rgb}{0.9, 0.17, 0.31}
\definecolor{ballblue}{rgb}{0.13, 0.67, 0.8}
\definecolor{onyx}{rgb}{0.06, 0.06, 0.06}


\lstset{
  breaklines=true,                 % automatic line breaking only at whitespace
  captionpos=b,                    % sets the caption-position to bottom
  breakatwhitespace=false,
  keepspaces=true,
  numbers=left,
  numbersep=5pt,
  showspaces=false,
  showstringspaces=false,
  showtabs=false,
  tabsize=4,  
  backgroundcolor=\color{white},   % choose the background color
  commentstyle=\color{ao},    % comment style
  keywordstyle=\color{amaranth},    % keyword style
  stringstyle=\color{blue-violet},    % string literal style
  numberstyle=\tiny\color{ballblue},	   % number style
  basicstyle=\ttfamily\footnotesize\color{onyx} % size of fonts used for the code
}

%Document Header
\title{\textbf{Department of CSE\\SSN College of Engineering}}
\author{\textbf{Vishakan Subramanian - 18 5001 196 - Semester VII}}
\date{20 October 2021}

\begin{document}
\maketitle
\hrule
\section*{\center{UCS 1712 - Graphics And Multimedia Lab}}
\hrule
\bigskip

%Assignment Details
\subsection*{\center{\textbf{Exercise 12: Creating A 2D Animation}}}
\subsection*{\flushleft{Aim:}}
\begin{flushleft}

\begin{enumerate}

\item Using GIMP, include layers and create a simple animation of your choice. 

\end{enumerate}
 
\end{flushleft}


%Input
\newpage
\subsection*{\flushleft{Input: Frame-1}}
\begin{figure}[h]
\centering
\caption{Frame-1.}
\includegraphics[height=3cm, width=12cm]{Frame-1.png}
\end{figure}

\subsection*{\flushleft{Input: Frame-2}}
\begin{figure}[h]
\centering
\caption{Frame-2.}
\includegraphics[height=3cm, width=12cm]{Frame-2.png}
\end{figure}

\subsection*{\flushleft{Input: Frame-3}}
\begin{figure}[h]
\centering
\caption{Frame-3.}
\includegraphics[height=3cm, width=12cm]{Frame-3.png}
\end{figure}

%Output
\newpage
\subsection*{\flushleft{Output:}}
\bigskip
The frames were animated sequentially and exported as a \textbf{GIF} image, with each frame having a delay of 1000ms. The animation was viewed through a GIF Viewer.



%Learning Outcome
\subsection*{\flushleft{Learning Outcome:}}
\begin{itemize}
\item I learnt how to create an empty banner image using GIMP.
\item I learnt how to implement a border around the image using another layer.
\item I learnt how to type in custom text and color it using the Text tool.
\item I learnt how to import an external image as a layer using GIMP.
\item I understood how to calculate delays for each frame and implement it in GIMP.
\item I learnt how to scale an object while maintaining its aspect ratio using the Scale tool.
\item I learnt how to animate the frames using the Filters $\rightarrow$ Animate filter.
\item I learnt how to loop the animation forever and optimize it for a GIF using GIMP.
\end{itemize}


\end{document}